%%%%%%%%%%%%%%%%%%%%%%%%%%%%%%%%%%%%%%%%%%%%%%%%%%%%%%%%%%%%
%%%%%%%%%%%%%%%%%%%%%%%%%%%%%%%%%%%%%%%%%%%%%%%%%%%%%%%%%%%%
%%%%%%%%%%%%%%%%%%%%%%%%%%%%%%%%%%%%%%%%%%%%%%%%%%%%%%%%%%%%
%%%%%%%%%%%%%%%%%%%%%%%%%%%%%%%%%%%%%%%%%%%%%%%%%%%%%%%%%%%%
%%%%%%%%%%%%%%%%%%%%%%%%%%%%%%%%%%%%%%%%%%%%%%%%%%%%%%%%%%%%
\documentclass[12pt]{article}
\usepackage{fancyhdr}
\usepackage{pslatex}
\usepackage{epsfig}
\usepackage{times}
\usepackage{amsmath}
\usepackage{mathrsfs}
\usepackage[dvipsnames]{xcolor}
\usepackage[hidelinks]{hyperref}%renewcommand{\topfraction}{1.0}
\renewcommand{\topfraction}{1.0}
\renewcommand{\bottomfraction}{1.0}
\renewcommand{\textfraction}{0.0}
\setlength {\textwidth}{6.6in}
\hoffset=-1.0in
\oddsidemargin=1.00in
\marginparsep=0.0in
\marginparwidth=0.0in                                                                               
\setlength {\textheight}{9.0in}
\voffset=-1.00in
\topmargin=1.0in
\headheight=0.0in
\headsep=0.00in
\footskip=0.50in                                         
\setcounter{page}{1}
\begin{document}
\def\pos{\medskip\quad}
\def\subpos{\smallskip \qquad}
\newfont{\nice}{cmr12 scaled 1250}
\newfont{\name}{cmr12 scaled 1080}
\newfont{\swell}{cmbx12 scaled 800}
%%%%%%%%%%%%%%%%%%%%%%%%%%%%%%%%%%%%%%%%%%%%%%%%%%%%%%%%%%%%
%     DO NOT CHANGE ANYTHING ABOVE THIS LINE
%%%%%%%%%%%%%%%%%%%%%%%%%%%%%%%%%%%%%%%%%%%%%%%%%%%%%%%%%%%%
%     DO NOT CHANGE ANYTHING ABOVE THIS LINE
%%%%%%%%%%%%%%%%%%%%%%%%%%%%%%%%%%%%%%%%%%%%%%%%%%%%%%%%%%%%
%     DO NOT CHANGE ANYTHING ABOVE THIS LINE
%%%%%%%%%%%%%%%%%%%%%%%%%%%%%%%%%%%%%%%%%%%%%%%%%%%%%%%%%%%%

\begin{center}
{\large
PHYS 20323/60323: Fall 2025 - LaTeX Example
}\\
%%%%%%%%%%%%%%%%%%%%%%%%%%%%%%%%%%%%%%%%%%%%%%%%%%%%%%%%%%%%

%%%%%%%%%%%%%%%%%%%%%%%%%%%%%%%%%%%%%%%%%%%%%%%%%%%%%%%%%%%%
\end{center}
\begin{enumerate}
    \item At time $t = 0$ a particle is represented by the wave function

    \[
    \Psi(x) =
    \begin{cases}
      A \dfrac{x}{a}, & 0 \le x \le a,\\[6pt]
      A \dfrac{(b - x)}{(b - a)}, & a \le x \le b,\\[6pt]
      0, & \text{otherwise.}
    \end{cases}
    \]

    where A, $a$, and $b$ are constants.

    \begin{enumerate}
        \item  (3.3 points) Normalize $\Psi$ (i.e., find $A$ terms of $a$ and $b$).
        \item (3.3 points) Where is the particle likely to be found at $t = 0$?.
        \item[(c)] (3.4 points) What is the expectation value of $x$?.
    \end{enumerate}
\item \textbf{The following questions refer to stars in the Table below.}\\
\textit{Note: There may be multiple answers.}

\begin{center}
\begin{tabular}{|c | c | c | c | c | c | c | }
\hline
Name  & Mass & Luminosity & Lifetime & Temperature & Radius & Variable ? \\ \hline
\delta \text{ Scu.} & 2.0 M_\odot & & 5.0 \times 10^8 \text{ years} & & 2.0 R_\odot & Y \\ \hline
\gamma \text{ Del.} & 0.7 M_\odot & & 4.5 \times 10^{10}  \text{ years} & 5000 K & & N \\ \hline
\beta \text{ Cyg.} & 1.3 M_\odot & 3.5 L_\odot & & & & Y \\ \hline
\eta \text{ Car.} & 60. M_\odot & 10^6 L_\odot & 8.0 \times 10^5 \text{ years}  &  & & Y \\ \hline
\epsilon \text{ Eri.} & 6.0 M_\odot & 10^3 L_\odot & & 20,000 K & & N \\ \hline
\alpha \text{ Cen.} & 1.0 M_\odot & & & 6000 K & 1.0 R_\odot & N \\ \hline
\end{tabular}
\end{center}

\begin{enumerate}
    \item (4 points) Which of these stars will produce a planetary nebula.
    \item (4 points) Elements heavier than \texttt{Carbon} will be produced in which stars.
\end{enumerate}

    \item An electron is found to be in the spin state (in the $z$-basis): \chi = A \begin{pmatrix}
3i\\
4
\end{pmatrix}


\begin{enumerate}
        \item (5 points) Determine the possible values of A such that the state is normalized.
        \item (5 points) Find the expectation values of the operators \(\color{red}{S_x}\), \(\color{BrickRed}{S_y}\), \(\color{orange}{S_z}\) and  \(\vec{S}^{\,2}\).
    \end{enumerate}


\indent The matrix representations in the $z$-basis for the components of electron spin operators are given by:

\[
\textcolor{red}{\mathbf{S_x}} \textcolor{red}{= \frac{\hbar}{2}}
\textcolor{red}{
\begin{pmatrix}
0 & 1 \\
1 & 0
\end{pmatrix}}
\; \color{BrickRed}{;} \qquad
\color{BrickRed}{\mathbf{S_y}}
\color{BrickRed}{= \frac{\hbar}{2}}
\color{BrickRed}{
\begin{pmatrix}
0 & -i \\
i & 0
\end{pmatrix}}}
\; \color{Orange}{;} \qquad
\textcolor{orange}{\mathbf{S_z}}
= \frac{\hbar}{2}
\textcolor{orange}{
\begin{pmatrix}
1 & 0 \\
0 & -1
\end{pmatrix}}
\]



\end{enumerate}


\end{document}
